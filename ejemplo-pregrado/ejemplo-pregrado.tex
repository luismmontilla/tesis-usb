\documentclass[pregrado]{tesis-usb}

% paquetes
\usepackage[utf8]{inputenc}
\usepackage{verbatim}
\usepackage{acronym}
\usepackage{amsmath}
\usepackage{amsfonts}
\usepackage{amssymb}
% \usepackage{hyperref}

% estilo de las referencias
\bibliographystyle{plain}

\autor{Contreras-Sajo-Castelli}
\autori{N. Apellido}
\usbid{99-99999}
\titulo{Ejemplo de clase tesis-usb-cls}
\fecha{Mayo~de~2015}
\agno{2015}
\fechadefensa{31~de~abril~de~2015}
\tutor{Nombre y Apellido}
% \usarcotutor
\cotutor{Nombre y Apellido \mbox{(Afiliaci\'on)}} 
\trabajo{Tipo de Trabajo}
\coord{Nombre de Coordinaci\'on}
\grado{Grado del Programa}
\carrera{Carrera}
\programa{Nombre del Programa}
\juradouno{Nombre y Apellido}
\juradodos{Nombre y Apellido \mbox{(Afiliaci\'on)}}
\juradotres{Nombre y Apellido}

% Cambia comillas simple por comilla cerrada en ambiente verbatim 
\makeatletter
\let \@sverbatim \@verbatim
\def \@verbatim {\@sverbatim \verbatimplus}
{\catcode`'=13 \gdef \verbatimplus{\catcode`'=13 \chardef '=13 }} 
\makeatother

\begin{document}

\frontmatter
\maketitle
\input{dedicatoria}
\input{agradecimientos}
\begin{resumen}
     Es una exposici\'on clara del tema tratado en el trabajo, de los objetivos, de la metodolog\'ia utilizada, de los resultados relevantes obtenidos y de las conclusiones. Mismo tipo de fuente seleccionado con tamaño 12 e interlineado sencillo en el p\'arrafo. El resumen no debe exceder de trescientas (300) palabras escritas. \\
     Palabras cl\'aves: palabras, cl\'aves, separadas por coma, cinco m\'aximo.
\end{resumen}
\tableofcontents
\listoffigures
\listoftables
\useacronyms
\input{notacion}

\mainmatter
\input{introduccion}
\input{usodelaclase}
\chapter{SOBRE EL USO DE ACR\'ONIMOS Y LA LISTA DE S\'IMBOLOS}
\section{Acr\'onimos}
En este cap\'itulo se describe una forma de crear los acr\'onimos compatiblemente con las normas de los decanatos. Su uso es opcional pero recomendado.  

El uso de los acronimos se hace a trav\'es del paquete {\sl acronym} (que debe ser cargado en el prea\'ambulo) y es habilitado
con el comando \verb+\useacronyms+ al principio del archivo (luego de los \'indices durante \verb+\frontmatter+). Los
acr\'onimos se definen {\em todos} en el archivo {\tt acronimos.tex} bajo el ambiente \verb+\begin{acronym}+\verb+\end{acronym}+. Para definir un acr\'onimo se usa el comando \verb+\acro{}[]{}+. La primera entrada es el nombre del acr\'onimo, la segunda entrada es el acr\'onimo propiamente y la tercera entrada es la expansi\'on del acr\'onimo. Por ejemplo, lo siguiente es el contenido del archivo \texttt{acronimos.tex} donde se crean algunas definiciones de acr\'onimos.
\begin{verbatim}
\chapter*{LISTA DE ACR\'ONIMOS}
\begin{acronym}
\acro{USB}[USB]{Universidad Sim\'on Bol\'ivar}
\acro{CCE}[Dpto.~CCE]{Departamento de C\'omputo Cient\'ifico 
     y Estad\'isitica}
\acro{DEP}[DEP]{Decanato de Estudios Profesionales}
\acro{PDF}[PDF]{Documento en Formato Portable\copyright}
\acro{PS}[PS]{PostScript\copyright}
\end{acronym}
\end{verbatim}
Una vez definidos los acr\'onimos se pueden referenciar con el comando \verb+\ac{}+ usando el nombre definido para el acr\'onimo. Por
ejemplo, \verb+\ac{USB}+ producir\'a \ac{USB}, mientras que \verb+\ac{CCE}+ producir\'a \ac{CCE}.

La clase {\sl tesis-usb} en acorde a las disposiciones del \ac{DEP}, despliega la descripci\'on
de cada acr\'onimo una sola vez cuando es referenciada la primera vez en cada cap\'itulo. Todo
los usos sucesivos no son expandidos, por ejemplo:

\noindent \verb+\ac{LI}+$\rightarrow$ \ac{LI}\\
\verb+\ac{LI}+$\rightarrow$ \ac{LI}\\
\verb+\ac{LI}+$\rightarrow$ \ac{LI}\\
\verb+\ac{LI}+$\rightarrow$ \ac{LI}\\
\verb+\ac{CCE}+$\rightarrow$ \ac{CCE}

Nota: Si el comando \verb+\useacronyms+ est\'a comentado al principio de {\tt main.tex} y se
usan acr\'onimos a lo largo del libro, estos no van a funcionar resultando en errores de compilación.

\section{Lista de s\'imbolos}
La lista de s\'imbolos o notaci\'on matem\'atica se recomienda hacer manualmente. Por ejemplo, se puede incluir el siguiente c\'odigo luego de los \'indices.
\begin{verbatim}
\chapter*{Notaci\'on matem\'atica}
\begin{tabular}{ll}
$\mathbb{R}$ & Conjunto de n\'umeros reales\\
$M_{m,n}$ & Espacio de las matrices de tama\~no $m$ por 
     $n$ con entradas reales\\
$\mathcal{L}$ & Operador de Laplace\\
$\emptyset$ & Conjunto vac\'io
\end{tabular}
\end{verbatim}
\chapter{DE C\'OMO COMPILAR EL LIBRO}
Para compilar el libro, siendo \texttt{main.tex} el documento principal, se requieren los siguientes pasos (no siempre son todos necesarios)
\begin{enumerate}
\item \verb+~ latex main.tex+
\item \verb+~ bibtex main.aux+
\item \verb+~ latex main.tex+
\item \verb+~ latex main.tex+
\end{enumerate}
Si adicionalmente se quiere una salida de \ac{PDF}, entonces los siguiente pasos adicionales son
necesarios
\begin{enumerate}
\item \verb+~ dvips main.dvi+
\item \verb+~ ps2pdf main.ps+
\end{enumerate}
Finalmente, la salida en \ac{PDF} es {\tt main.pdf}

Algunos editores de \LaTeX\ efectuan todos los pasos anterior de manera automatica, sin embargo,
para que las referencias de hiperv\'inculos dentro del documento final \ac{PDF} funcionen,
es necesario primero compilar a \ac{PS} primero.

Aternativamente se puede compilar directamente a \ac{PDF} usado \verb+pdflatex+
\begin{enumerate}
\item \verb+~ pdflatex main.tex+
\item \verb+~ bibtex main.aux+
\item \verb+~ pdflatex main.tex+
\item \verb+~ pdflatex main.tex+
\end{enumerate}
\input{conclusiones}
\nocite{*}
\bibliography{referencias}
\appendix
\input{apendice-a.tex}
\chapter{EJEMPLO DE IMAGEN EN APÉNDICE}\label{apx:img}
\begin{figure}[ht]
\includegraphics[scale=0.48,angle=0]{ejemplo}
\caption[Título corto de imagen]{Título corto de imagen}\label{img:imgscl}
\end{figure}




\end{document}
